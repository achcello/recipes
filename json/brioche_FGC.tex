\documentclass[12pt]{article}
\usepackage[margin=1in,landscape]{geometry}
\usepackage{multicol}
\author{Alexis Gabriel Ainouz}
\title{Brioche}
\date{}
\begin{document}
\begin{multicols*}{2}
\maketitle

Source: https://www.youtube.com/watch?v=hY19NK7qBCU

\section{Notes}

Alex's brioche. This recipe was transcribed from the Youtube video by me, Aryn Harmon. Of the first four times I've made brioche, the first was excellent, and the next two were uncooked in the center. For the fourth, I lowered the temperature to 350F and extended the cooking time to about 35 minutes.

\section{Ingredients}

\begin{itemize}
    \item 500g baking flour
    \item 2 eggs + 1 for egg wash
    \item 50g sugar
    \item 8-10g salt
    \item 10g dry yeast (one packet works)
    \item 100g butter (about one 8 tbsp. stick)
    \item 200mL milk + 2 tbsp. for egg wash
\end{itemize}

\section{Method}

\begin{enumerate}
    \item Beat together the flour, eggs, milk, and yeast.
    \item Add the butter, sugar, salt; mix and knead until the dough forms a smooth, elastic ball (best done with a stand mixer, but it can be done by hand).
    \item Proof 1-2 hours, or until doubled.
    \item Push down the dough. Divide and roll into three long logs, each with a diameter between one and one and a half inches.
    \item Braid the three logs together, then place them in a lightly buttered loaf tin.
    \item Cover for a second proof, about 20 minutes, or until it just rises out of the loaf tin.
    \item Brush on the egg wash, then bake for about 25 minutes at 390F. (See notes for updates to cooking time)
    \item Let the loaf cool for a couple minutes.
\end{enumerate}

\end{multicols*}
\end{document}
